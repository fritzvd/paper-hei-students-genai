% Options for packages loaded elsewhere
% Options for packages loaded elsewhere
\PassOptionsToPackage{unicode}{hyperref}
\PassOptionsToPackage{hyphens}{url}
\PassOptionsToPackage{dvipsnames,svgnames,x11names}{xcolor}
%
\documentclass[
  letterpaper,
  DIV=11,
  numbers=noendperiod]{scrartcl}
\usepackage{xcolor}
\usepackage{amsmath,amssymb}
\setcounter{secnumdepth}{-\maxdimen} % remove section numbering
\usepackage{iftex}
\ifPDFTeX
  \usepackage[T1]{fontenc}
  \usepackage[utf8]{inputenc}
  \usepackage{textcomp} % provide euro and other symbols
\else % if luatex or xetex
  \usepackage{unicode-math} % this also loads fontspec
  \defaultfontfeatures{Scale=MatchLowercase}
  \defaultfontfeatures[\rmfamily]{Ligatures=TeX,Scale=1}
\fi
\usepackage{lmodern}
\ifPDFTeX\else
  % xetex/luatex font selection
\fi
% Use upquote if available, for straight quotes in verbatim environments
\IfFileExists{upquote.sty}{\usepackage{upquote}}{}
\IfFileExists{microtype.sty}{% use microtype if available
  \usepackage[]{microtype}
  \UseMicrotypeSet[protrusion]{basicmath} % disable protrusion for tt fonts
}{}
\makeatletter
\@ifundefined{KOMAClassName}{% if non-KOMA class
  \IfFileExists{parskip.sty}{%
    \usepackage{parskip}
  }{% else
    \setlength{\parindent}{0pt}
    \setlength{\parskip}{6pt plus 2pt minus 1pt}}
}{% if KOMA class
  \KOMAoptions{parskip=half}}
\makeatother
% Make \paragraph and \subparagraph free-standing
\makeatletter
\ifx\paragraph\undefined\else
  \let\oldparagraph\paragraph
  \renewcommand{\paragraph}{
    \@ifstar
      \xxxParagraphStar
      \xxxParagraphNoStar
  }
  \newcommand{\xxxParagraphStar}[1]{\oldparagraph*{#1}\mbox{}}
  \newcommand{\xxxParagraphNoStar}[1]{\oldparagraph{#1}\mbox{}}
\fi
\ifx\subparagraph\undefined\else
  \let\oldsubparagraph\subparagraph
  \renewcommand{\subparagraph}{
    \@ifstar
      \xxxSubParagraphStar
      \xxxSubParagraphNoStar
  }
  \newcommand{\xxxSubParagraphStar}[1]{\oldsubparagraph*{#1}\mbox{}}
  \newcommand{\xxxSubParagraphNoStar}[1]{\oldsubparagraph{#1}\mbox{}}
\fi
\makeatother


\usepackage{longtable,booktabs,array}
\usepackage{calc} % for calculating minipage widths
% Correct order of tables after \paragraph or \subparagraph
\usepackage{etoolbox}
\makeatletter
\patchcmd\longtable{\par}{\if@noskipsec\mbox{}\fi\par}{}{}
\makeatother
% Allow footnotes in longtable head/foot
\IfFileExists{footnotehyper.sty}{\usepackage{footnotehyper}}{\usepackage{footnote}}
\makesavenoteenv{longtable}
\usepackage{graphicx}
\makeatletter
\newsavebox\pandoc@box
\newcommand*\pandocbounded[1]{% scales image to fit in text height/width
  \sbox\pandoc@box{#1}%
  \Gscale@div\@tempa{\textheight}{\dimexpr\ht\pandoc@box+\dp\pandoc@box\relax}%
  \Gscale@div\@tempb{\linewidth}{\wd\pandoc@box}%
  \ifdim\@tempb\p@<\@tempa\p@\let\@tempa\@tempb\fi% select the smaller of both
  \ifdim\@tempa\p@<\p@\scalebox{\@tempa}{\usebox\pandoc@box}%
  \else\usebox{\pandoc@box}%
  \fi%
}
% Set default figure placement to htbp
\def\fps@figure{htbp}
\makeatother


% definitions for citeproc citations
\NewDocumentCommand\citeproctext{}{}
\NewDocumentCommand\citeproc{mm}{%
  \begingroup\def\citeproctext{#2}\cite{#1}\endgroup}
\makeatletter
 % allow citations to break across lines
 \let\@cite@ofmt\@firstofone
 % avoid brackets around text for \cite:
 \def\@biblabel#1{}
 \def\@cite#1#2{{#1\if@tempswa , #2\fi}}
\makeatother
\newlength{\cslhangindent}
\setlength{\cslhangindent}{1.5em}
\newlength{\csllabelwidth}
\setlength{\csllabelwidth}{3em}
\newenvironment{CSLReferences}[2] % #1 hanging-indent, #2 entry-spacing
 {\begin{list}{}{%
  \setlength{\itemindent}{0pt}
  \setlength{\leftmargin}{0pt}
  \setlength{\parsep}{0pt}
  % turn on hanging indent if param 1 is 1
  \ifodd #1
   \setlength{\leftmargin}{\cslhangindent}
   \setlength{\itemindent}{-1\cslhangindent}
  \fi
  % set entry spacing
  \setlength{\itemsep}{#2\baselineskip}}}
 {\end{list}}
\usepackage{calc}
\newcommand{\CSLBlock}[1]{\hfill\break\parbox[t]{\linewidth}{\strut\ignorespaces#1\strut}}
\newcommand{\CSLLeftMargin}[1]{\parbox[t]{\csllabelwidth}{\strut#1\strut}}
\newcommand{\CSLRightInline}[1]{\parbox[t]{\linewidth - \csllabelwidth}{\strut#1\strut}}
\newcommand{\CSLIndent}[1]{\hspace{\cslhangindent}#1}



\setlength{\emergencystretch}{3em} % prevent overfull lines

\providecommand{\tightlist}{%
  \setlength{\itemsep}{0pt}\setlength{\parskip}{0pt}}



 


\KOMAoption{captions}{tableheading}
\makeatletter
\@ifpackageloaded{caption}{}{\usepackage{caption}}
\AtBeginDocument{%
\ifdefined\contentsname
  \renewcommand*\contentsname{Table of contents}
\else
  \newcommand\contentsname{Table of contents}
\fi
\ifdefined\listfigurename
  \renewcommand*\listfigurename{List of Figures}
\else
  \newcommand\listfigurename{List of Figures}
\fi
\ifdefined\listtablename
  \renewcommand*\listtablename{List of Tables}
\else
  \newcommand\listtablename{List of Tables}
\fi
\ifdefined\figurename
  \renewcommand*\figurename{Figure}
\else
  \newcommand\figurename{Figure}
\fi
\ifdefined\tablename
  \renewcommand*\tablename{Table}
\else
  \newcommand\tablename{Table}
\fi
}
\@ifpackageloaded{float}{}{\usepackage{float}}
\floatstyle{ruled}
\@ifundefined{c@chapter}{\newfloat{codelisting}{h}{lop}}{\newfloat{codelisting}{h}{lop}[chapter]}
\floatname{codelisting}{Listing}
\newcommand*\listoflistings{\listof{codelisting}{List of Listings}}
\makeatother
\makeatletter
\makeatother
\makeatletter
\@ifpackageloaded{caption}{}{\usepackage{caption}}
\@ifpackageloaded{subcaption}{}{\usepackage{subcaption}}
\makeatother
\usepackage{bookmark}
\IfFileExists{xurl.sty}{\usepackage{xurl}}{} % add URL line breaks if available
\urlstyle{same}
\hypersetup{
  pdftitle={Patterns of Myth, Over-promise and Hype in New Technologies: A case of Generative AI amongst Higher Education students.},
  pdfkeywords={technological innovation, beliefs, generative artificial
intelligence},
  colorlinks=true,
  linkcolor={blue},
  filecolor={Maroon},
  citecolor={Blue},
  urlcolor={Blue},
  pdfcreator={LaTeX via pandoc}}


\title{Patterns of Myth, Over-promise and Hype in New Technologies: A
case of Generative AI amongst Higher Education students.}
\author{Fritz Van Deventer}
\date{}
\begin{document}
\maketitle
\begin{abstract}
Technological innovation, such as Generative Artificial Intelligence
(genAI), is often promoted with hyperbole and exaggeration. Misleading
beliefs about its capabilities arise and are hard to decouple from the
actual utility it can bring. This pattern paper aims to identify key
traits of unrealistic beliefs, so students in Higher Education can make
more informed decisions on how to approach technological innovations,
such as genAI.
\end{abstract}


\subsection{Introduction}\label{introduction}

\emph{``I think in many ways GPT-5 is already smarter than me and many
other people,''} says Sam Altman when interviewed by Die Welt {[}1{]}.
Anthromorphisation, mystification and hype currently have surrounded
Artificial Intelligence since the late 1950s {[}2{]}. AI is not alone in
being communicated in a ``hyped'' fashion, that can involve
``exaggerations about the significance or certainty of research findings
{[}3{]}. This has been the case for a lot of topics such as but not
limited to: neuroimaging, stem cell research, nanotechnology, genomics,
and artifical intelligence.

Over the last few years there has been a great wave of AI applications,
or more accurately software applications bundled with generative AI
(GenAI) amongst other forms of AI. GenAI uses a variety of models to
generate text, images, music and even video. Online platforms like
ChatGPT, but also Copilot which is now embedded in the whole Microsoft
Office suite, and even plain Google Search which always comes with a
generated answer on top of the page. Signified with ``AI'' or ✨ it is
usually a form of a Large Language Model that interprets requests and
gives human-like answers.

The permeation of AI in our tools and systems is coupled with this
belief that if we do not participate, we might miss the boat and will
not be able to keep up with the renewed productivity our AI-using
colleagues and competitors will enjoy.

But what we are seeing is not new. Firstly we have seen this before with
the exact same term in different guises since the 1950s: AI {[}2{]}.
What we are seeing is a repetition of technology industry overpromising
new techniques or technologies and packaging them as a panacea for all
kinds of problems or General Purpose Tools, while also decreasing
emphasis on a large variety of negative consequences that this
technology also brings. For example the profound physical impact these
new technologies are having on our environment {[}4{]}, the security
issues of models with jailbreak attempts {[}7{]}. Technologies are not
neutral by nature, and do not automatically point toward ``progress''.

This paper puts forward a pattern that helps recognizing language and
rhetoric surrounding technological innovations to deflate the hype for
students in Higher Education Institutes (HEIs).

\subsection{Related work}\label{related-work}

Surrounding ``hype'' a lot of research has already been done concerning
studying hype {[}8{]}, or in its problems and defining it as something
other than scientific misconduct {[}3{]}, also concerning the role of
emotional and/or logical expectations {[}9{]} and comparing different
hypes and their life cycles {[}10{]}.

Myth and technology have a long history together. Collectively we adhere
to the idea that with technology and change we also converge to a better
reality, to progress. {[}11{]} states that believing in this myth is a
religious act itself.

We project ideas of what a future society can look like with these new
technologies which can be called \emph{sociotechnic imaginaries}
{[}12{]}.

This paper seeks to find a pattern to recognize the process where
mythical belief surrounding technological innovation creates myth and
polarizes criticism into categories of Luddism and technophilia, while
societal pressure to participate increases.

\subsection{Methodology}\label{methodology}

This research will involve two types of qualitative analysis of language
(and rhetoric) used in GenAI discourse and news articles as well as
language used by HEI students.

\subsubsection{Thematic analysis}\label{thematic-analysis}

\emph{Thematic analysis} will be used as {[}13{]} and {[}14{]} describe
it to find common themes and topics in personal interview with students.
Thematic analysis is a more subjective approach to data, where language
is central. It requires a \emph{coding} process that uses semantic and
latent codes to categorize statements and find \emph{themes} and
\emph{subthemes} of meaning in texts.

\subsubsection{Critical Discourse
Analysis}\label{critical-discourse-analysis}

Through \emph{critical discourse analyis} (CDA) I will look at what kind
of language is constructing narratives, describes relations of power and
how it creates or reinforces myths {[}15{]}.

CDA tries to find patterns in power dynamics and social effects in
discourse. In this analysis we can take the texts further and analyze
context and intertextuality to challenge social inequalities. I will
focus mainly on step two: \emph{Identify obstacles to addressing the
social wrong} of this process which consists of three stages {[}16{]}:

\begin{enumerate}
\def\labelenumi{\arabic{enumi}.}
\tightlist
\item
  Analyse dialectical relations
\item
  Select texts
\item
  Analyse texts interdiscursively and linguistically
\end{enumerate}

\subsection{Problem}\label{problem}

The problem with new technologies is the ``hype'' created around it that
leads to false beliefs or ``creeds'', which are hard to decouple from
what the technologies actually bring to the table.

\subsection{Solution}\label{solution}

By discovering themes and power relations in language we can describe a
recurring phenomenon in different technological transformations.

\includegraphics[width=3.88in,height=7.28in]{paper-imaginaries-hype-myth_files/figure-latex/mermaid-figure-2.png}

\textsubscript{Source:
\href{https://fritzvd.github.io/paper-hei-students-genai/paper-imaginaries-hype-myth.qmd.html}{Article
Notebook}}

\subsection{Consequences}\label{consequences}

Being able to recognize language as hyperbolic or mythical empowers
students to discern truth from propaganda.

\subsection{Forces}\label{forces}

\subsubsection{Solutions over needs}\label{solutions-over-needs}

Instead of looking at our needs we are often distracted by technology
and try to find a way to fit in our workflow {[}17{]}.

\subsubsection{Pace of change and fear of missing
out}\label{pace-of-change-and-fear-of-missing-out}

Technological innovations happen quickly and people fall prey to
innovation anxiety {[}18{]}. When anxious it is harder to discern truth
from fiction.

\subsubsection{Advertisement from
companies}\label{advertisement-from-companies}

Industry leaders have a very real interest in hyping their own tool,
model, progress or company as they have financial interests in
attrracting more users and perhaps ruling out competition. Terms like
inevitability of the technology combined with solutionism approaches to
a wide-scala of problems are good examples of ``myth''-smells {[}19{]}.

\subsection*{References}\label{references}
\addcontentsline{toc}{subsection}{References}

\phantomsection\label{refs}
\begin{CSLReferences}{0}{0}
\bibitem[\citeproctext]{ref-burgardOpenAIChefSamAltman2025}
\CSLLeftMargin{1. }%
\CSLRightInline{Burgard, J.P.: {OpenAI-Chef Sam Altman}: ,,{Ich} glaube
nicht, dass die {KI Menschen} wie {Ameisen} behandeln wird`` - {WELT},
(2025).}

\bibitem[\citeproctext]{ref-guestUncriticalAdoptionAI2025}
\CSLLeftMargin{2. }%
\CSLRightInline{Guest, O., Suarez, M., Müller, B., van Meerkerk, E.,
Oude Groote Beverborg, A., de Haan, R., Reyes Elizondo, A., Blokpoel,
M., Scharfenberg, N., Kleinherenbrink, A., Camerino, I., Woensdregt, M.,
Monett, D., Brown, J., Avraamidou, L., Alenda-Demoutiez, J., Hermans,
F., van Rooij, I.: Against the {Uncritical Adoption} of '{AI}'
{Technologies} in {Academia}, (2025).
https://doi.org/\href{https://doi.org/10.5281/ZENODO.17065099}{10.5281/ZENODO.17065099}.}

\bibitem[\citeproctext]{ref-intemannUnderstandingProblemHype2022}
\CSLLeftMargin{3. }%
\CSLRightInline{Intemann, K.: Understanding the {Problem} of
{``{Hype}''}: {Exaggeration}, {Values}, and {Trust} in {Science}.
Canadian Journal of Philosophy. 52, 279--294 (2022).
https://doi.org/\href{https://doi.org/10.1017/can.2020.45}{10.1017/can.2020.45}.}

\bibitem[\citeproctext]{ref-crawford2021atlas}
\CSLLeftMargin{4. }%
\CSLRightInline{Crawford, K.: The atlas of {AI}: {Power}, politics, and
the planetary costs of artificial intelligence. Yale University Press
(2021).}

\bibitem[\citeproctext]{ref-perezIgnorePreviousPrompt2022}
\CSLLeftMargin{5. }%
\CSLRightInline{Perez, F., Ribeiro, I.: Ignore {Previous Prompt}:
{Attack Techniques For Language Models},
\url{https://arxiv.org/abs/2211.09527}, (2022).
https://doi.org/\href{https://doi.org/10.48550/arXiv.2211.09527}{10.48550/arXiv.2211.09527}.}

\bibitem[\citeproctext]{ref-liMultistepJailbreakingPrivacy2023}
\CSLLeftMargin{6. }%
\CSLRightInline{Li, H., Guo, D., Fan, W., Xu, M., Huang, J., Meng, F.,
Song, Y.: Multi-step {Jailbreaking Privacy Attacks} on {ChatGPT},
\url{https://arxiv.org/abs/2304.05197}, (2023).
https://doi.org/\href{https://doi.org/10.48550/arXiv.2304.05197}{10.48550/arXiv.2304.05197}.}

\bibitem[\citeproctext]{ref-carliniExtractingTrainingData2021}
\CSLLeftMargin{7. }%
\CSLRightInline{Carlini, N., Tramer, F., Wallace, E., Jagielski, M.,
Herbert-Voss, A., Lee, K., Roberts, A., Brown, T., Song, D., Erlingsson,
U., Oprea, A., Raffel, C.: Extracting {Training Data} from {Large
Language Models}, \url{https://arxiv.org/abs/2012.07805}, (2021).
https://doi.org/\href{https://doi.org/10.48550/arXiv.2012.07805}{10.48550/arXiv.2012.07805}.}

\bibitem[\citeproctext]{ref-dedehayirHypeCycleModel2016}
\CSLLeftMargin{8. }%
\CSLRightInline{Dedehayir, O., Steinert, M.: The hype cycle model: {A}
review and future directions. Technological Forecasting and Social
Change. 108, 28--41 (2016).
https://doi.org/\href{https://doi.org/10.1016/j.techfore.2016.04.005}{10.1016/j.techfore.2016.04.005}.}

\bibitem[\citeproctext]{ref-shiRoleExpectationInnovation2023}
\CSLLeftMargin{9. }%
\CSLRightInline{Shi, Y., Herniman, J.: The role of expectation in
innovation evolution: {Exploring} hype cycles. Technovation. 119, 102459
(2023).
https://doi.org/\href{https://doi.org/10.1016/j.technovation.2022.102459}{10.1016/j.technovation.2022.102459}.}

\bibitem[\citeproctext]{ref-vanlenteComparingTechnologicalHype2013}
\CSLLeftMargin{10. }%
\CSLRightInline{van Lente, H., Spitters, C., Peine, A.: Comparing
technological hype cycles: {Towards} a theory. Technological Forecasting
and Social Change. 80, 1615--1628 (2013).
https://doi.org/\href{https://doi.org/10.1016/j.techfore.2012.12.004}{10.1016/j.techfore.2012.12.004}.}

\bibitem[\citeproctext]{ref-burdett2014religion}
\CSLLeftMargin{11. }%
\CSLRightInline{Burdett, M.S.: The religion of technology:
{Transhumanism} and the myth of progress. Religion and Transhumanism:
The Unknown Future of Human Enhancement. 131 (2014).}

\bibitem[\citeproctext]{ref-jasanoffContainingAtomSociotechnical2009}
\CSLLeftMargin{12. }%
\CSLRightInline{Jasanoff, S., Kim, S.-H.: Containing the {Atom}:
{Sociotechnical Imaginaries} and {Nuclear Power} in the {United States}
and {South Korea}. Minerva. 47, 119--146 (2009).
https://doi.org/\href{https://doi.org/10.1007/s11024-009-9124-4}{10.1007/s11024-009-9124-4}.}

\bibitem[\citeproctext]{ref-naeemStepbyStepProcessThematic2023}
\CSLLeftMargin{13. }%
\CSLRightInline{Naeem, M., Ozuem, W., Howell, K., Ranfagni, S.: A
{Step-by-Step Process} of {Thematic Analysis} to {Develop} a {Conceptual
Model} in {Qualitative Research}. International Journal of Qualitative
Methods. 22, 16094069231205789 (2023).
https://doi.org/\href{https://doi.org/10.1177/16094069231205789}{10.1177/16094069231205789}.}

\bibitem[\citeproctext]{ref-braun2021thematic}
\CSLLeftMargin{14. }%
\CSLRightInline{Braun, V., Clarke, V.: Thematic analysis: A practical
guide to understanding and doing. Thousand Oaks. (2021).}

\bibitem[\citeproctext]{ref-blommaertCriticalDiscourseAnalysis2000}
\CSLLeftMargin{15. }%
\CSLRightInline{Blommaert, J., Bulcaen, C.:
\href{https://www.jstor.org/stable/223428}{Critical {Discourse
Analysis}}. Annual Review of Anthropology. 29, 447--466 (2000).}

\bibitem[\citeproctext]{ref-fairclough2013critical}
\CSLLeftMargin{16. }%
\CSLRightInline{Fairclough, N.: Critical discourse analysis: {The}
critical study of language. Routledge (2013).}

\bibitem[\citeproctext]{ref-ervinCanTechnologyFulfill1993}
\CSLLeftMargin{17. }%
\CSLRightInline{Ervin, G.L.: Can {Technology Fulfill Its Promise}? IALLT
Journal of Language Learning Technologies. 26, 7--16 (1993).
https://doi.org/\href{https://doi.org/10.17161/iallt.v26i2.9498}{10.17161/iallt.v26i2.9498}.}

\bibitem[\citeproctext]{ref-okerekeInnovationAnxietyNew}
\CSLLeftMargin{18. }%
\CSLRightInline{Okereke, C.: Innovation {Anxiety}: {The New Age
Stressor} \textbar{} {Chibs Okereke Stress} \& {Burnout Coach}.}

\bibitem[\citeproctext]{ref-zheng2025resisting}
\CSLLeftMargin{19. }%
\CSLRightInline{Zheng, K., Huber, L., Stark, A., Kim, N., Lameiro, F.,
Santo, W.L., Chowdhary, S., Kim, E., Zhang, J.:
\href{https://arxiv.org/abs/2508.08313}{Resisting {AI} solutionism
through workplace collective action}. arXiv preprint arXiv:2508.08313.
(2025).}

\end{CSLReferences}




\end{document}
